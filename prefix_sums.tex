\documentclass[]{article}

\usepackage[T2A]{fontenc}
\usepackage[utf8]{inputenc}
\usepackage[bulgarian,english]{babel}
\usepackage{amsmath, amsthm}

\usepackage{tikz, pgf}

\newtheorem{problem}{Задача}
\newenvironment{solution}{\noindent{\bf Решение.}\hspace*{1em}}{\qed\par}

\usepackage[cache=false]{minted}

%opening
\title{Префиксни суми}
\author{}

\begin{document}

\maketitle

\section{Загрявка}
\begin{problem}
Намерете броя числа в интервала $[l, r]$, които се делят на 7.\newline
Ограничения: $1\leq l\leq r\leq 10^{18}$.
\end{problem}
\begin{solution}
\newline
Едно решение е да обходим и проверим всички числа в интервала, но ще е доста бавно. Може да го забързаме като намерим първото число кратно на 7 и после прескачаме през 7, но пак може да се наложи да проверим много числа.\newline
Да разгледаме друг подход. Знаем, че всяко седмо число е кратно на седем. От там идва идеята да вземем броя числа в интервала и да го разделим на 7. Идеята е много добра, но не съвсем вярна - има интервали с по равен брой числа, но различен брой кратни на 7, например интервалите $[6,13]$ и $[7,14]$. Това, че всяко седмо число е кратно на 7, ще ни даде правилно решение, ако броя на числата в интервала е кратен на 7.
Може да използваме това, като проверим последните няколко числа поотделно, и да намалим интервала така, че той да стане с кратен на 7 брой числа и да използваме формула за останалия интервал. Например в интервала $[20,99]$ има 80 числа. За да получим кратен на 7 брой може да махнем последните 3. Така ще проверим за 99, 98 и 97, и ще използваме формула за интервала $[20,96]$. 98 e кратно на 7, в $[20,96]$ има $77/7=11$ кратни на 7, и общия отговор за интервала $[20,99]$ ще е 12.\newline
Друг подобен вариант е да намерим първото и последното число кратни на 7 и да използваме формула свързана с тях. Ако първото и последното число кратни на 7 са $a$ и $b$, то броят на всички е $(b-a)/7+1$.\newline
Последните два варианта решават задачата, но има доста случай за които трябва да се внимава. Също ако търсим кратни не на 7, а на нещо доста по-голямо ще трябва по-умно да намираме последното число в интервала кратно на даденото.\newline
Целта на тази глава е да ви накара винаги когато видите нещо, което се търси в произволен интервал, да пробвате да го разбиете на два интервала, които започват от едно място, например 0 или 1. В тази задача в сила е следната важна връзка - броя числа кратни на 7 в $[l,r]$ e равен на броя числа кратни на 7 в $[1,r]$ минус броя числа кратни на 7 в $[1,l-1]$. Иначе казано, ако преброим числата кратни на 7 от 1 до $r$, ще сме броили и тези по-малки от $l$, за това трябва да ги извадим. Важно е да отбележим, че интервала, който вадим е до $l-1$, понеже самото $l$, не трябва го вадим.\newline
Остава да видим дали да сметнем броя числа кратни на 7 в интервала $[1, n]$. Тук вече доста по-лесно да съобразим, че отговора е точно $n/7$.\newline
Следва имплементация на решението:
\begin{minted}{cpp}
long long solve(long long l, long long r) {
	return r%7-(l-1)%7;
}
\end{minted}
\end{solution}

\begin{problem}
Намерете броя числа в интервала $[l, r]$, които се делят на d.\newline
Ограничения: $1\leq l\leq r\leq 10^{18},1\leq d\leq 10^{18}$.
\end{problem}
\begin{solution}
Отново използваме разделянето на два интервала $[1, r]$ и $[1, l-1]$. След което може да обобщим, че броят числа кратни на $d$ в интервала $[1, n]$ е равен на $n/d$.\newline
\begin{minted}{cpp}
long long solve(long long l, long long r, long long d) {
    return r%d-(l-1)%d;
}
\end{minted}
\end{solution}


\section{Загрявка}
\begin{problem}
Намерете броя числа в интервала $[l, r]$, които се делят на 7.\newline
Ограничения: $1\leq l\leq r\leq 10^{18}$.
\end{problem}
\begin{solution}
Задачата има различни варианти за решение.\newline
Може да обходим и проверим всички числа в интервала, което ще е бавно в повечето случаи.\newline
За по-добро решение ще ни трябва нещо друго. Първата част изглежда по-лесна, да видим какво става ако намерим всички числа кратни на 3. От този брой ще трябва да извадим тези които сме преброили(кратни на 3) и се делят на 5 - това са числата кратни на 15. Така задачата се свежда до две подобни подзадачи - да намерим броя числа, които се делят на 3 и на 15.\newline
Нека сега да намерим броя числа в интервала $[l, r]$, които се делят на 3. Знаем, че всяко трето число е кратно на 3. От там идва идеята да вземем броя числа в интервала и да го разделим на 3. Идеята е много добра, но е грешна в интервала $[2,5]$ има 1 число, а в интервала $[3,6]$ - 2. Това, че всяко трето число е кратно на 3, ще ни даде правилно решение, ако броя на числата в интервала е кратен на 3. Това ни дава идея, че може да проверим последните 0,1 или 2 числа поотделно, и да използваме формула за останалия интервал. Например в интервала $[20,99]$ има 80 числа. За да получим кратен на 3 брой може да махнем последните 2. Така ще проверим за 99 и 98, и ще използваме формула за интервала $[20,97]$. 99 e кратно на 3, в $[20,97]$ има $78/3=26$ кратни на 3, и общия отговор за интервала $[20,99]$ ще е 27.\newline
Горното решение за намиране на броя числа кратни на 3 в интервал е достатъчно сложно да предразполага към грешки както в идеята, така и в писането на кода. За това ще разгледаме още едно. Целта тук е винаги когато видите нещо, което се търси в произволен интервал, да пробвате да го разбиете на два интервала, които започват от 0 или 1. В тази задача в сила е следната важна връзка - броя числа кратни на 3 в $[l,r]$ e равен на броя числа кратни на 3 в $[1,r]$ минус броя числа кратни на 3 в $[1,l-1]$. Иначе казано, ако преброим числата кратни на 3 в интервала $[1,r]$, ще сме броили и тези преди $l$, за това трябва да ги извадим. Важно е да отбележим, че интервала, който вадим е до $l-1$, важни е самото $l$, да не го вадим.\newline
Първото нещо важно го направихме - да се сетим да пробваме да решим задачата като гледаме интервали от началото. Остана второто - да видим дали може лесно да сметнем броя числа кратни на 3 в интервала $[1, n]$. Тук вече може по-лесно да съобразим, че отговора е $n/3$.\newline
Може да обобщим, че брой числа кратни на $d$ в интервала $[1, n]$ е равен на $n/d$.\newline
Следва имплементация на решението:
\begin{minted}{cpp}
long long solve(long long l, long long r) {
	long long multiples3 = r/3 - (l-1)/3;
	long long multiples3 = r/15 - (l-1)/15;
	return multiples3-multiples15;
}
\end{minted}
\end{solution}

\section{Префиксни суми}

\begin{problem}
Даден е масив $a$ с $n$ числа и $q$ заявки. За всяка заявка са дадени две числа $l$ и $r$, и трябва да намерите сумата на подмасива $[l, r]$, т.е. $a[l]+a[l+1]+...+a[r]$.\newline
Ограничения: $0\leq n\leq 1000000,0\leq q\leq 1000000,0 \leq a[i]\leq 1000000$.
\end{problem}

\begin{solution}
Първо ще отбележим, че сумата на числата може да стане голяма и за това ще я пазим в променлива от тип $long\ long$.\newline
Задачата има очевидно решение - за всяка заявка обхождаме всички числа $a[l],a[l+1],...,a[r]$ и ги събираме. Това решение обаче не e за максимален резултат, понеже е бавно. За всяка заявка може да са необходими близо до $n$ събирания, като умножим по $q$ заявки, получаваме $nq$ операции. Това е доста голямо число при дадените ограничения.\newline
Идеята за подобрение идва от това, че работим с интервали. Първото правило при решаване на задачи с интервали $[a,b]$ е да проверим дали може да решим задачата


\end{solution}



$p[i]=p[i-1]+a[i]$
$n=p_1^{x_1}.p_2^{x_2}...p_k^{x_k}$
\begin{equation} 
S (\omega)=\frac{\alpha g^2}{\omega^5} \,
e ^{[-0.74\bigl\{\frac{\omega U_\omega 19.5}{g}\bigr\}^{-4}]}
\end{equation}

\begin{tikzpicture}
\fill[gray!20](0,0) rectangle (2.5,2.5); 
\fill[blue](0,1) rectangle (2,2.5);
\draw[step=0.5cm] (0,0) grid (2.5,2.5);
\end{tikzpicture}
\begin{tikzpicture}
\fill[gray!20](0,0) rectangle (2.5,2.5); 
\fill[red](0,1) rectangle (1.5,2.5);
\draw[step=0.5cm] (0,0) grid (2.5,2.5);
\end{tikzpicture}
\begin{tikzpicture}
\fill[gray!20](0,0) rectangle (2.5,2.5); 
\fill[green](0,1.5) rectangle (2,2.5);
\draw[step=0.5cm] (0,0) grid (2.5,2.5);
\end{tikzpicture}
\begin{tikzpicture}
\fill[gray!20](0,0) rectangle (2.5,2.5); 
\fill[yellow](0,1.5) rectangle (1.5,2.5);
\draw[step=0.5cm] (0,0) grid (2.5,2.5);
\end{tikzpicture}
\begin{tikzpicture}
\fill[gray!20](0,0) rectangle (2.5,2.5); 
\fill[blue](1.5,1) rectangle (2,1.5);
\fill[red](0,1) rectangle (1.5,2.5);
\fill[green](0,1.5) rectangle (2,2.5);
\fill[yellow](0,1.5) rectangle (1.5,2.5);
\draw[step=0.5cm] (0,0) grid (2.5,2.5);
\end{tikzpicture}
\newline
Префикса сума е сума от началото до някъде.
Най-лесно когато мислим за двумерен масив, е да си представяме таблица с редове и колони. За да създадем таблица с N реда и M колони използваме $type\ name[N][M]$, където $type$ е типът на данните, които ще съхраняваме в таблицата - $bool$, $int$, $long\ long$, $char$, $string$, ..., а $name$ - името на таблицата.\newline
Например може да създадем таблица, в която ще пазим цели числа по следния начин - $int\ table[3][4]$. Тя ще има 3 реда и 4 колони, и ще изглежда ето така:\newline

\end{document}
